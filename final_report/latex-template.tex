\documentclass[autodetect-engine,dvi=dvipdfmx,ja=standard,
               a4j,11pt]{bxjsarticle}

\RequirePackage{geometry}
\geometry{reset,paperwidth=210truemm,paperheight=297truemm}
\geometry{hmargin=.75truein,top=20truemm,bottom=25truemm,footskip=10truemm,headheight=0mm}
%\geometry{showframe} % 本文の"枠"を確認したければ,コメントアウト
\usepackage{graphicx}
\usepackage{fancyvrb}
\usepackage{spverbatim}
\renewcommand{\theFancyVerbLine}{\texttt{\footnotesize{\arabic{FancyVerbLine}:}}}

\title{コンパイラ実験}

\author{学生番号: 09B54923549\\
        中嶋 空偉 (NAKAJIMA, Sorai)}
\date{\number\year 年\number\month 月\number\day 日}

%%======== 本文 ====================================================%%
\begin{document}
\maketitle
% 目次つきの表紙ページにする場合はコメントを外す
%{\footnotesize \tableofcontents \newpage}

%--------------------------------------------------------------------%
\section{実験の目的} \label{sec:abstract}

\section{言語の定義}

\section{受理されるプログラム例}

\section{コード生成}
% メモリの使い方
% レジスタの使い方
% 算術式のコード生成の方法
% 工夫した点 変数の格納かな

\section{最終課題のプログラムと実行結果}

\subsection{1から10までの数の和}

\subsubsection{プログラム}

\begin{Verbatim}[numbers=left, xleftmargin=10mm, numbersep=6pt,
                 fontsize=\small, baselinestretch=0.8]
define i;
define sum;
sum = 0;
i = 1;
while(i < 11) {
  sum = sum + i;
  i = i + 1;
}
\end{Verbatim}

\subsubsection{実行結果}
% 実行結果とstep数

\subsection{5の階乗}

\subsubsection{プログラム}

\begin{Verbatim}[numbers=left, xleftmargin=10mm, numbersep=6pt,
                 fontsize=\small, baselinestretch=0.8]
define i;
define fact;

fact = 1;
i = 1;
while(i < 6) {
  fact = fact * i;
  i = i + 1;
}
fact;
\end{Verbatim}

\subsubsection{実行結果}
% 実行結果とstep数

\subsection{FizzBuzz}

\subsubsection{プログラム}

\begin{Verbatim}[numbers=left, xleftmargin=10mm, numbersep=6pt,
                 fontsize=\small, baselinestretch=0.8]
define fizz;
define buzz;
define fizzbuzz;
define others;
define i;
fizz = 0;
buzz = 0;
fizzbuzz = 0;
others = 0;
i = 1;
while(i < 31) {
  if((i / 15) * 15 == i) {
    fizzbuzz = fizzbuzz + 1;
  } else {
    if((i / 3) * 3 == i) {
      fizz = fizz + 1;
    } else {
      if((i / 5) * 5 == i) {
        buzz = buzz + 1;
      } else {
        others = others + 1;
      }
    }
  }
  i = i + 1;
}
others;
\end{Verbatim}

\subsubsection{実行結果}
% 実行結果とstep数

\section{考察}
% レポート書いていく中で難しかった部分をメモしておいて考察とする.


% \begin{thebibliography}{99}
%  \bibitem{book:algodata} 平田富雄,アルゴリズムとデータ構造,森北出版,1990.
%  \bibitem{book:label2} 著者名,書名,出版社,発行年.
%  \bibitem{www:label3} WWWページタイトル,\spverb!https://example.of.too.long.url.jp/you.must.use.spverb/and.insert. a.space.somewhere.to.avoid.overfull.html!,アクセス日.
% \end{thebibliography}
%
%--------------------------------------------------------------------%
\end{document}
