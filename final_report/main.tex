\documentclass[autodetect-engine,dvi=dvipdfmx,ja=standard,
               a4j,11pt]{bxjsarticle}

\RequirePackage{geometry}
\geometry{reset,paperwidth=210truemm,paperheight=297truemm}
\geometry{hmargin=.75truein,top=20truemm,bottom=25truemm,footskip=10truemm,headheight=0mm}
%\geometry{showframe} % 本文の"枠"を確認したければ,コメントアウト
\usepackage{graphicx}
\usepackage{fancyvrb}
\usepackage{spverbatim}
\renewcommand{\theFancyVerbLine}{\texttt{\footnotesize{\arabic{FancyVerbLine}:}}}

\title{コンパイラ実験}

\author{学生番号: 09B54923549\\
        中嶋 空偉 (NAKAJIMA, Sorai)}
\date{\number\year 年\number\month 月\number\day 日}

%%======== 本文 ====================================================%%
\begin{document}
\maketitle
% 目次つきの表紙ページにする場合はコメントを外す
%{\footnotesize \tableofcontents \newpage}

%--------------------------------------------------------------------%
\section{実験の概要と目的} \label{sec:abstract}
\subsection{本実験の概要}
この実験では,独自に定義した言語をソース言語として,アセンブリ言語を
目的言語として変換するコンパイラを作成する.
コンパイラ作成の際には,字句解析を行うプログラムを生成するものとしてflex,構文解析を行うプログラムを生成するものとしてbison,astとコード生成に関してはc言語でプログラムを作成する.

\subsection{本実験の目的}
3年間の総仕上げとしてある程度大きなプログラムを作成する.
仕様を決め,プログラムを書き,データ構造を決め,アルゴリズムを考え,テストパターンを決め,デバッグするなどの一連の流れを経験する.


\section{言語の定義}

本コンパイラが受理する言語の文法定義を以下に示す.
これはBisonの構文規則からC言語アクションコードを取り除いたものである.

\begin{Verbatim}[numbers=left, xleftmargin=10mm, numbersep=6pt,
                 fontsize=\footnotesize, baselinestretch=0.9]
program
    : declarations statements
;

declarations
    : decl_statement declarations
    | decl_statement
;

decl_statement
    : DEFINE IDENT SEMIC
    | ARRAY IDENT L_BRACKET NUMBER R_BRACKET SEMIC
    | ARRAY IDENT L_BRACKET NUMBER R_BRACKET L_BRACKET NUMBER R_BRACKET SEMIC
;

statements
    : statement statements
    | statement
;

statement
    : assignment_stmt
    | loop_stmt
    | cond_stmt
    | expression SEMIC
;

assignment_stmt
    : IDENT ASSIGN expression SEMIC
    | IDENT L_BRACKET expression R_BRACKET ASSIGN expression SEMIC
    | IDENT L_BRACKET expression R_BRACKET L_BRACKET expression R_BRACKET ASSIGN expression SEMIC
;

expression
    : expression add_op term
    | term
;

term
    : term mul_op factor
    | factor
;

factor
    : var
    | NUMBER
    | L_PAREN expression R_PAREN
    | IDENT L_BRACKET expression R_BRACKET
    | IDENT L_BRACKET expression R_BRACKET L_BRACKET expression R_BRACKET
;

add_op
    : PLUS
    | MINUS
;

mul_op
    : MUL
    | DIV
;

var
    : IDENT
;

loop_stmt
    : WHILE L_PAREN condition R_PAREN L_BRACE statements R_BRACE
;

cond_stmt
    : IF L_PAREN condition R_PAREN L_BRACE statements R_BRACE ELSE L_BRACE statements R_BRACE
    | IF L_PAREN condition R_PAREN L_BRACE statements R_BRACE
;

condition
    : expression cond_op expression
;

cond_op
    : EQ
    | NE
    | LE
    | GE
    | LT
    | GT
;
\end{Verbatim}

\section{受理されるプログラム例}

本コンパイラで受理できるプログラムの例を示す.

\subsection{基本演算とWhileループ}

まず,基本的な演算と繰り返し処理の例として,1から10までの和を計算するプログラムを示す.

\begin{Verbatim}[numbers=left, xleftmargin=10mm, numbersep=6pt,
                 fontsize=\small, baselinestretch=0.9]
define i;
define sum;
sum = 0;
i = 1;
while (i < 11) {
  sum = sum + i;
  i = i + 1;
}
sum;
\end{Verbatim}

このプログラムでは,変数\verb|i|を1から10まで1ずつ増やしながら,\verb|sum|に加算していくことで和を求めている.

\subsection{条件分岐 (If-Else)}

次に,条件分岐の例として,2つの数値の大小比較を行い,大きい方から小さい方を減算するプログラムを示す.

\begin{Verbatim}[numbers=left, xleftmargin=10mm, numbersep=6pt,
                 fontsize=\small, baselinestretch=0.9]
define a;
define b;
a = 10;
b = 5;
if (a > b) {
  a = a - b;
} else {
  a = a + b;
}
a;
\end{Verbatim}

このプログラムでは,\verb|if-else|文を使用して条件によって実行する処理を分岐している.本コンパイラでは,\verb|==|,\verb|!=|,\verb|<=|,\verb|>=|,\verb|<|,\verb|>|の比較演算子が利用可能である.

\subsection{配列の使用}

最後に,配列の例として,3行3列の2次元配列の対角成分の和を計算するプログラムを示す.

\begin{Verbatim}[numbers=left, xleftmargin=10mm, numbersep=6pt,
                 fontsize=\small, baselinestretch=0.9]
array arr[3][3];
define sum;
define i;
arr[0][0] = 1;
arr[0][1] = 2;
arr[0][2] = 3;
arr[1][0] = 4;
arr[1][1] = 5;
arr[1][2] = 6;
arr[2][0] = 7;
arr[2][1] = 8;
arr[2][2] = 9;
sum = 0;
i = 0;
while (i < 3) {
  sum = sum + arr[i][i];
  i = i + 1;
}
sum;
\end{Verbatim}

このプログラムでは,2次元配列を宣言し,対角成分(\verb|arr[0][0]|,\verb|arr[1][1]|,\verb|arr[2][2]|)を\verb|while|ループを用いて順に足し合わせている.1次元配列と2次元配列の両方がサポートされている.

\section{コード生成}
% メモリの使い方
% レジスタの使い方
% 基本的には,引数はv0に格納.さらに必要になればまたメモリに格納を繰り返す実装とした.
% 算術式のコード生成の方法
% どんな実装にしたっけ.(振り返り必要)
% 他にも主要なif文, 条件文などは章立てて説明していきたい.
\section{特に工夫した点: 変数の管理方法}
% 他の学生はプログラム上にラベルとして埋め込む人が多かった中,スタックとしてメモリに格納することに決めた.
% 変数名をキーとしてoffsetを取得するdictionary?にしたっけ(要確認).
% なぜこうしたかというと他の言語を勉強した際に,スコープごとにスタックに変数を格納していく形だったので,直感でこの形で進めることに決めた.
% この実装については思い出すために質問で深ぼる必要あり.結構前に作ったため.

\section{最終課題のプログラムと実行結果}

\subsection{1から10までの数の和}

\subsubsection{プログラム}

\begin{Verbatim}[numbers=left, xleftmargin=10mm, numbersep=6pt,
                 fontsize=\small, baselinestretch=0.8]
define i;
define sum;
sum = 0;
i = 1;
while(i < 11) {
  sum = sum + i;
  i = i + 1;
}
\end{Verbatim}

\subsubsection{実行結果}
% 実行結果とstep数

\subsection{5の階乗}

\subsubsection{プログラム}

\begin{Verbatim}[numbers=left, xleftmargin=10mm, numbersep=6pt,
                 fontsize=\small, baselinestretch=0.8]
define i;
define fact;

fact = 1;
i = 1;
while(i < 6) {
  fact = fact * i;
  i = i + 1;
}
fact;
\end{Verbatim}

\subsubsection{実行結果}
% 実行結果とstep数

\subsection{FizzBuzz}

\subsubsection{プログラム}

\begin{Verbatim}[numbers=left, xleftmargin=10mm, numbersep=6pt,
                 fontsize=\small, baselinestretch=0.8]
define fizz;
define buzz;
define fizzbuzz;
define others;
define i;
fizz = 0;
buzz = 0;
fizzbuzz = 0;
others = 0;
i = 1;
while(i < 31) {
  if((i / 15) * 15 == i) {
    fizzbuzz = fizzbuzz + 1;
  } else {
    if((i / 3) * 3 == i) {
      fizz = fizz + 1;
    } else {
      if((i / 5) * 5 == i) {
        buzz = buzz + 1;
      } else {
        others = others + 1;
      }
    }
  }
  i = i + 1;
}
others;
\end{Verbatim}

\subsubsection{実行結果}
% 実行結果とstep数

\section{考察}
% レポート書いていく中で難しかった部分をメモしておいて考察とする.


% \begin{thebibliography}{99}
%  \bibitem{book:algodata} 平田富雄,アルゴリズムとデータ構造,森北出版,1990.
%  \bibitem{book:label2} 著者名,書名,出版社,発行年.
%  \bibitem{www:label3} WWWページタイトル,\spverb!https://example.of.too.long.url.jp/you.must.use.spverb/and.insert. a.space.somewhere.to.avoid.overfull.html!,アクセス日.
% \end{thebibliography}
%
%--------------------------------------------------------------------%
\end{document}
